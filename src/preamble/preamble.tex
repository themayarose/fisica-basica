% Configuração de Texto
\usepackage{fontspec}
\usepackage{microtype}
\usepackage{lipsum}

\edef\restoreparindent{\parindent=\the\parindent\relax}
\usepackage{parskip}
\restoreparindent

% Tradução
\usepackage{polyglossia}
\setdefaultlanguage{brazil}
\usepackage[pt-BR]{datetime2}

% Imagens
\usepackage{graphicx}
\graphicspath{{imagens/},{graficos/},{diagramas/}}

% Unidades Sistema Internacional
\usepackage[cdot,squaren]{SIunits}
\usepackage{subfigure}
\providecommand\subfigureautorefname{Figura}

% Cores
\usepackage[usenames, dvipsnames]{color}
\usepackage[table]{xcolor}
\usepackage{tabularx}

% Layout do documento
\usepackage[hmarginratio=3:2,vmarginratio=3:2,top=30mm,left=30mm]{geometry}
\usepackage[hang, small,labelfont=bf,up,textfont=it,up]{caption}
\usepackage{float}
\usepackage{balance}
\usepackage{placeins}
\renewcommand{\baselinestretch}{1.5}

% Tabelas
\usepackage{booktabs}

% Referências
\usepackage[hidelinks,brazil]{hyperref}
\usepackage[language=brazil,style=abnt,indent]{biblatex}
\usepackage{csquotes}

% Matemática
\usepackage{amsmath}
\usepackage{amssymb}
\usepackage{braket}
\usepackage{mathtools}
\usepackage{amsthm}
\usepackage{thmtools}

\numberwithin{equation}{subsection}
\numberwithin{figure}{subsection}
\numberwithin{table}{subsection}

\declaretheoremstyle[bodyfont=\normalfont\small,spaceabove=10pt, %
    spacebelow=10pt]{egstyle}
\declaretheoremstyle[bodyfont=\normalfont,spaceabove=10pt, %
    spacebelow=10pt,notefont=\normalfont\bfseries,notebraces={- }{}]{eqstyle}
\declaretheorem[name=Exemplo,parent=subsection,style=egstyle]{eg}
\declaretheorem[name=Equação,sibling=equation,style=eqstyle]{namedeq}

% Primeira letra maior
\usepackage{lettrine}

% Customização do resumo
\usepackage[addtotoc]{abstract}
\renewcommand{\abstractnamefont}{\normalfont\bfseries}
\renewcommand{\abstracttextfont}{\normalfont\itshape}

% Apêndices
\usepackage[titletoc,title]{appendix}
\usepackage{etoolbox, apptools}
\AtBeginEnvironment{appendices}{\appendixtrue}

% Customização dos intertítulos
\usepackage{titlesec}
\renewcommand\thesection{\Roman{section}}
\renewcommand\thesubsection{\thesection.\arabic{subsection}}
\renewcommand\thesubsubsection{\thesubsection.\alph{subsubsection})}
\titleformat{\section}[block]{\huge\bfseries\scshape\centering}{\IfAppendix{\appendixname~}{\relax}{\thesection.~}\IfAppendix{:~}{}}{1em}{}
\titleformat{\subsection}[block]{\large\bfseries}{\thesubsection.~}{1em}{}
\titlespacing*{\subsection}{0pt}{10pt}{10pt}
\newcommand{\sectionbreak}{\clearpage}


% Customização de cabeçalho e rodapé
\usepackage{fancyhdr}
\pagestyle{fancy}
\fancyhead{}
\fancyfoot{}
\renewcommand{\headrulewidth}{0pt}
% \fancyhead[C]{Publicado em \DTMlangsetup{showdayofmonth=false}\today\DTMlangsetup{showdayofmonth=true}} % Custom header text
\fancyfoot[RO,LE]{\thepage}

% Fonte
\setromanfont{Libre Baskerville}
\linespread{1.1}
\usepackage{euscript}[mathcal]
\usepackage{commath}

% Código-fonte
\usepackage{listings}
\setmonofont{Fira Code}
\lstset{
    basicstyle=\ttfamily\small,
    columns=fullflexible,keepspaces,
    inputencoding=utf8
}
\newcommand*{\Apêndiceautorefname}{Apêndice}

\addto\captionsbrazil{%
    \renewcommand{\lstlistingname}{Código}
    \renewcommand{\lstlistlistingname}{Lista de \lstlistingname s}
    \renewcommand{\appendixname}{Apêndice}
    \renewcommand{\appendixpagename}{Apêndices}
}

