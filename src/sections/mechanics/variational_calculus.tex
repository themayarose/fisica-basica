\subsection{Cálculo variacional}
\label{ssec:variational_calculus}

Seja $\mathcal{F}(q_1(t), \ldots, q_n(t), \dot q_1(t), \ldots, \dot q_n(t)) :=
\int_{t_0}^{t_1} \dif t~f(t, q_1(t), \ldots, q_n(t), \dot q_1(t), \ldots, \dot
q_n(t))$  um funcional que possua mínimos locais nas funções $\mathcal{Q} :=
\left\{\chi_1(t), \ldots, \chi_n(t) \right\}$. Então, $\forall i \in \{1,
\ldots,  n\}$, $\mathcal{Q}$ é a solução do sistema de equações diferenciais:

\begin{equation}
    \label{eq:lagrange}
    \dod{}{t} \del{ \dpd{f}{\dot q_i} } = \dpd{f}{q_i}
\end{equation}

\begin{eg}[Princípio de Fermat]
    \label{eg:fermat}
    O princípio de Fermat diz que a luz andando num meio percorre o caminho que
    minimiza o \textbf{tempo} de percurso. Isto é, dado um meio bidimensional
    cujo índice de refração depende da posição ($n = n(x,y)$), temos:

    \begin{equation}
        \begin{aligned}
        T = \int_{t_0}^{t_1} \dif t =&  \\
        = \frac{1}{c} \int_{t_0}^{t_1} \dif t~\frac{c}{v} \dod{s}{t} =& \\
        = \frac{1}{c} \int_{A}^{B} \dif s~n(x,y)  =& \\
        = \frac{1}{c} \int_{A}^{B} \sqrt{{\dif x}^2 + {\dif y}^2} n(x,y) =& \\
        = \frac{1}{c} \int_{x_0}^{x_1} \dif x \sqrt{1 + {\dot y}^2} ~ n(x,y) 
        \end{aligned}
    \end{equation}

    Onde $\dot y := \tod{y}{x}$. Desse modo, se definirmos o funcional
    $\mathcal{T}(y, \dot y) := \int_{x_0}^{x_1} \dif x \sqrt{1 + {\dot y}^2} ~
    n(x,y)$,  sabemos que o caminho $y(x)$ é solução da \autoref{eq:lagrange}
    para $f(x, y, \dot y) = \sqrt{1 + {\dot y}^2}~n(x,y)$.
\end{eg}


\begin{eg}[Catenária]
    \label{eg:catenaria}
    A catenária é a curva que minimiza a energia potencial gravitacional de uma
    corda inelástica presa pelas suas duas extremidades, e cujo corpo é livre e
    não encosta no chão.

    A energia potencial gravitacional de uma partícula puntiforme é dada por
    $E_g = mgy$, e, considerando uma corda com densidade linear de massa $\rho$,
    podemos fazer:

    \begin{equation}
        \begin{aligned}
            E_g = \int_M \dif m ~ gy =& \\
            = \int_A^B \dif s~\rho gy =& \\
            = \rho g \int_A^B \sqrt{{\dif x}^2 + {\dif y}^2}~y =& \\
            = \rho g \int_{x_0}^{x_1} \dif x~y \sqrt{1 + {\dot y}^2}
        \end{aligned}
    \end{equation}

    Novamente, $\dot y := \tod{y}{x}$. Também de forma análoga ao
    \autoref{eg:fermat}, definindo o funcional $\mathcal{E}(y, \dot y) :=
    \int_{x_0}^{x_1} \dif x~y \sqrt{1 + {\dot y}^2}$, teremos que a curva $y(x)$
    será a catenária, e será solução da \autoref{eq:lagrange} para $f(y, \dot y)
    = y \sqrt{1 + {\dot y}^2}$.
\end{eg}

Entre outros exemplos úteis, temos:

\begin{table}[H]
	\centering
	\begin{tabularx}{\textwidth}{lXr}
		\toprule
		Nome & Definição & Equação \\
        \midrule
        Braquistócrona & %
            Superfície que minimiza o tempo que uma partícula demora para %
            deslizar sob influência de um campo gravitacional & %
            $f(y, \dot y) = y^{\frac{1}{2}}~\sqrt{1 + {\dot y}^2}$ \\ 
        Geodésica hiperbólica & %
            Menor caminho entre dois pontos em um semiplano hiperbólico & %
            $f(y, \dot y) = y^{-1}~\sqrt{1 + {\dot y}^2}$ \\
		\bottomrule
	\end{tabularx}
	\caption{Resultados comuns de cálculos variacionais}
	\label{tab:varcalc_eg}
\end{table}

Essas equações podem ser derivadas de maneira extremamente similar à do
\autoref{eg:fermat} e do \autoref{eg:catenaria}. 

De maneira geral, se um funcional tem um lagrangiano (i.e. $\mathcal{F}=\int
\dif t~L$) independente do tempo (no caso, a variável de integração), pode-se
usar a Identidade de Beltrami para encontrar grandezas constantes que auxiliam
a resolução das equações de Euler-Lagrange:

\begin{namedeq}[Identidade de Beltrami]
    \begin{equation*}
        \label{eq:beltrami}
        \dod{}{t} \del{L - \dot q_i \dpd{L}{\dot q_i}} = 0
    \end{equation*}
\end{namedeq}