\subsection{Equações de Lagrange e de Hamilton}
\label{ssec:lagrange_hamilton}

\subsubsection{Por que $L = T - V$?}
\label{sssec:lagrange_eq}

Do princípio de d'Alembert\footnote{O princípio de d'Alembert diz que toda
    partícula num sistema, independentemente do número de forças concretas que
    atuem sobre ela, na verdade pode ser interpretado como estando em
    \textit{equilíbrio exceto pela ação de uma força externa}, que causa
movimento (ainda que virtual).  Ou seja, para cada partícula indexada por $i \in
\mathbb{N}$, $\vec F_i - \dot{\vec p}_i = 0$.}, pode-se chegar a uma
\textbf{força generalizada}, em função das coordenadas generalizadas, que é dada
por

\begin{equation*}
    Q_k = \sum_{i=1}^N \vec F_i \dpd{\vec r_i}{q_k} \text{.}
\end{equation*}

Utilizando-se do princípio de d'Alembert, de cálculo variacional e após diversos
passos, pode-se chegar a:

\begin{equation}
    \label{eq:lagrange_original}
    \dod{}{t} \del{\dpd{T}{\dot q_k}} - \dpd{T}{q_k} = Q_k \text{.}
\end{equation}

Quando a força $Q_k$ deriva de um potencial escalar, i.e., $\vec F_i = -
\nabla_i V$, temos que

\begin{equation*}
    Q_k = - \dpd{V}{q_k}
\end{equation*}

e se $Q_k = Q_k(q_{1k}(t), \ldots, q_{nk}(t), t)$, isto é, $Q_k$ não depende da
velocidade da partícula $k$, sabemos que

\begin{equation*}
    \dpd{Q_k}{\dot q_k} = 0 ~\Rightarrow~ \dod{}{t} \del{\dpd{Q_k}{\dot q_k}} =
    0
\end{equation*}

e daí, com $L := T - V$,

\begin{equation*}
    \dod{}{t} \del{\dpd{T}{\dot q_k}} - \dpd{T}{q_k} = Q_k
    ~\Rightarrow~
    \dod{}{t} \del{\dpd{L}{\dot q_k}} - \dpd{L}{q_k} = 0 \text{.}
\end{equation*}

Para forças que dependam da velocidade (e.g. força de Lorentz), deve-se utilizar
o potencial generalizado $U$, definido como

\begin{equation*}
    Q_k = - \dpd{U}{q_k} + \dod{}{t} \del{\dpd{U}{\dot q_k}} \text{.}
\end{equation*}

Essa definição alternativa do potencial permite manter válida a
\autoref{eq:lagrange}, para $L = T - U$, e, por exemplo, no caso da força de
Lorentz, leva ao potencial

\begin{equation*}
    U = q \del{\varphi - {\vec v} \cdot {\vec A}} \text{,}
\end{equation*}

onde $\vec A \equiv \text{potencial vetor}$, $\varphi \equiv \text{potencial
elétrico}$, $q \equiv \text{carga elétrica}$ e $\vec v \equiv \text{velocidade
da partícula}$.

\subsubsection{Equações de Hamilton}
\label{sssec:hamilton}

Pode-se definir a hamiltoniana de um sistema a partir da transformada de
Legendre\footnote{Transformadas de Legendre também serão extremamente úteis no
estudo das funções termodinâmicas, na \autoref{ssec:thermodynamical_functions}.}
da sua lagrangiana.


